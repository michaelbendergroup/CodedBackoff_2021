\appendix

\section{Probability Inequalities}

\begin{lemma}\label{lemma:BinomialPoisson}
  Let $X$ be a binomial random variable and $Y$ be a Poisson random
  variable such that $\E[X]=\E[Y]=\mu$ for some $\mu>0$. Then,
  $\Pr(X=x)< \Pr(Y=x)$ for $x=0$ and all integers $x\geq 2\mu+1$.
\end{lemma}
\begin{proof}
  Since $X$ is a binomial random variable, we know that
  $X\sim\mathrm{Bin}(n,\mu/n)$ for some integer $n\geq 1$. Let's first
  look at $x=0$, where we have
  \[
  \Pr(X=0)=\left(1-\frac{\mu}{n}\right)^n < e^{-\mu} = \Pr(Y=0).
  \]
  For $x>0$, we have
  \begin{eqnarray*}
    \Pr(X=x) & = & 
        \binom{n}{x}\cdot \left(\frac{\mu}{n}\right)^x\cdot
        \left(1-\frac{\mu}{n}\right)^{n-x}\\
    & = & \left(\prod_{i=0}^{x-1}\frac{n-i}{n}\right)\cdot \frac{n^x}{x!}\cdot
        \left(\frac{\mu}{n}\right)^x\cdot
        \left(1-\frac{\mu}{n}\right)^{n} \cdot
        \left(1-\frac{\mu}{n}\right)^{-x}\\
    & < & \frac{\mu^x e^{-\mu}}{x!}\cdot
        \prod_{i=0}^{x-1}\left(1-\frac{i}{n}\right)\cdot
        \left(1-\frac{\mu}{n}\right)^{-x}\\
    & < & \Pr(Y=x)\cdot e^{\frac{x\mu}{n}-\sum_{i=0}^{x-1}\frac{i}{n}}\ =\ 
        \Pr(Y=x)\cdot e^{\frac{1}{n}\cdot\left(x\mu - \binom{x}{2}\right)}.
  \end{eqnarray*}
  We therefore have $\Pr(X=x)<\Pr(Y=x)$ if $\binom{x}{2}\geq x\mu$,
  and therefore if $x\geq 2\mu+1$.
\end{proof}

\begin{lemma}\label{lemma:PoissonShift}
  Let $X$ and $Y$ be two Poisson random variables with expectations $\E[X]=\lambda$ and $\E[Y]=(1+\delta)\lambda$ for some $\lambda>0$ and $0\leq \delta\leq \frac{\sqrt{5}-1}{2}$. For $k\geq 2\lambda$, it holds that
  \[
  \Pr(Y\geq k) \leq \frac{(1+\delta)^{k+2}}{e^{\delta\lambda}}\cdot \Pr(X\geq k).
  \]
\end{lemma}
\begin{proof}
  For every $i\geq 0$, we have
  \begin{equation}\label{eq:singlePoBound}
      \Pr(Y = i) = \frac{(1+\delta)^i\lambda^i}{i!}e^{-(1+\delta)\lambda} =
      (1+\delta)^i\cdot e^{-\delta \lambda}\cdot \Pr(X = i).
  \end{equation}
  Because $k>2\lambda$, for every $i\geq 0$, we further have
  \begin{equation}\label{eq:relativeProb}
      \Pr(X = k + i) = \frac{\lambda^{k+i}}{(k+i)!}\cdot e^{-\lambda} \leq
      \left(\frac{\lambda}{k}\right)^i\cdot \frac{\lambda^k}{k!}\cdot e^{-\lambda} =
      \left(\frac{1}{2}\right)^i\cdot \Pr(X = k).
  \end{equation}
  We can now bound the probability $\Pr(Y\geq k)$ as follows.
  \begin{eqnarray*}
      \Pr(Y \geq k) & \stackrel{\eqref{eq:singlePoBound}}{=} &
      \frac{1}{e^{\delta\lambda}}\cdot \sum_{i=k}^\infty (1+\delta)^i\cdot \Pr(X=i)\\
      & = & \frac{1}{e^{\delta\lambda}}\cdot\left[
      (1+\delta)^k\cdot \Pr(X\geq k) + \sum_{i=1}^\infty 
      \left((1+\delta)^{k+i}-(1+\delta)^k\right)\cdot \Pr(X=k+i)
      \right]\\
      & \stackrel{\eqref{eq:relativeProb}}{\leq}& \frac{(1+\delta)^k}{e^{\delta\lambda}}\cdot\left[
      \Pr(X\geq k) + \sum_{i=1}^\infty 
      \left(\left(\frac{1+\delta}{2}\right)^{i}-\frac{1}{2^i}\right)\cdot \Pr(X=k)
      \right]\\
      & \leq & \frac{(1+\delta)^k}{e^{\delta\lambda}}\cdot
      \frac{1+\delta}{1-\delta}\cdot \Pr(X\geq k)
      \ \stackrel{\big(\delta\leq \frac{\sqrt{5}-1}{2}\big)}{\leq}\ 
      \frac{(1+\delta)^{k+2}}{e^{\delta\lambda}}\cdot \Pr(X\geq k).
  \end{eqnarray*}
\end{proof}

We also use the following straightforward bounds on the tail probabilities of the sum of independent Bernoulli random variables.

\begin{lemma}\label{lemma:tailBounds}
  Let $X_i$, $i=1,\dots,n$, be independent Bernoulli random variables with $\Pr(X_i=1)=p_i$. Further, let $X:=\sum_{i=1}^n X_i$ and define $\mu:=\E[X]=\sum_{i=1}^n p_i$. We have
  \[
  \Pr(X=0) \leq e^{-\mu}\quad\text{and}\quad
  \forall k\geq \mu \, : \, \Pr(X > k)\leq \left(\frac{e \mu}{k+1}\right)^{k+1}\cdot e^{-\mu}.
  \]
\end{lemma}
\begin{proof}
  The first bound follows because
  \[
  \Pr(X=0) = \prod_{i=1}^n (1-p_i) \leq e^{-\sum_{i=1}^n p_i} = e^{-\mu}.
  \]
  The second bound follows from a standard Chernoff bound. For $\delta\geq 0$, we know that
  \[
  \Pr(X\geq (1+\delta)\mu)\leq \left(\frac{e^\delta}{(1+\delta)^{1+\delta}}\right)^\mu.
  \]
  The bound now follows by observing that $\Pr(X>k)=\Pr(X\geq k+1)$, by setting $\delta = (k+1)/\mu - 1$, and by using that $k\geq \mu$ and therefore $\delta\geq 0$.
\end{proof}