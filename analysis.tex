\section{Analysis}

\newcommand{\pmin}{p_{min}}

Throughout the execution, we will maintain a potential function $\phi(t)$ that is a function of the state of the system at time $t$.  This potential function will capture how much progress the system is making at delivering packets.

For each packet $i$, we define $p_i$ to be the joining probability of $i$.  We define the contention $c(t) = \sum_i p_i$.  We define $\pmin(t) = \min_i p_i$ to be the minimum probability of any packet currently in the system.

The potential function consists of three components:
\begin{enumerate}
    \item $N(t)$ is the number of packets that have been injected in the system at time $t$.
    \item $logC(t) = \threshold\left|\log\left(\frac{c(t)}{\tau}\right)\right|$ is the log of the contention and indicates how far the contention is from the ideal point $\tau$, i.e., how many epochs of length $\threshold$ are needed to reach $\tau$. 
    \item $s(t) = \threshold\log\left(\frac{c(t)}{\pmin(t)}\right)$ is designed to compensate for changes in potential when packets exit the system.
\end{enumerate}

The key claims we need to prove are as follows.  For the purpose of this paragraph, assume that ``with high probability'' means ``with probability of error exponentially small in $\threshold$.''
\begin{itemize}
    \item When $c(t)$ is close to $\tau$, we have a successful epoch with high probability.  Assume the epoch is $r$ rounds long.  When this occurs, $N(t)$ decreases by at least $r$.  The second term $logC(t)$ may increase, but it will be compensate for by the decrease in $s(t)$ so that the next change when a packet completes is always a decrease of at least $r$.
    \item When $c(t)$ is bigger than $\tau$, then with high probability an epoch contains no silent rounds.  Assume the epoch is $r$ rounds long.  In this case, either $r$ packets completes, and the potential decreases by at least $r$ (as per the previous case), or every packet decreases its joining probability by a factor of $2$ (and $r = \threshold$).  In the latter case, $N(t)$ remains unchanged, $logC(t)$ decreases by $\threshold$, and $s(t)$ remains unchanged.
    \item When $c(t)$ is smaller than $\tau$, then with high probability an epoch either contains a single silent round or there is a decoding event.  Assume the epoch is $r$ rounds long.  In this case, either $r$ packets completes, and the potential decreases by at least $r$ (as per the first case), or every packet increases its joining probability by a factor of $(1 + 1/\threshold)$ (and $r = 1$).  In the latter case, $N(t)$ remains unchanged, $logC(t)$ increases by $(1 + 1/\threshold)$, and $s(t)$ remains unchanged.
\end{itemize}

Let $A(t)$ be the number of packets that arrive at the beginning of round $t$.  With the above claims proved, we can show that, in expectation, the change in potential in every round is $A(t) - (1 - e^{-\Theta(\threshold)})$.

\subsection{So what?}

So we have shown that the potential function ensures that in each round, the expected decrease in potential is $\geq 1 - e^{-\theta(\threshold)}$.  What does that imply?

There seems to be a fundamental trade-off in terms of throughput:
\begin{itemize}
    \item If the initial probability $p$ is small, then if we inject only one packet, it will take $\theta(\threshold \log(1/p)$ rounds for it to broadcast successfully.  
    \item If the initial probability $p$ is larger, then consider the following unfortunate execution: the adversary inject $4(\threshold/p)$ packets every $\threshold$ rounds (with the likelihood each time that they do not succeed, because contention is $> \threshold$).  This continues for $pn/\threshold$ epochs for $pn$ rounds until there are $n$ nodes in the system.  Assume, then, that the $n$ packets now all finish in the next $n$ rounds.  Then the achieved throughput is $n$ packets in $n(1 + p)$ rounds.  So unless $p$ is subconstant, e.g., $e^{-\threshold}$, we are not going to get the throughput we want.
\end{itemize}
Note that this feels like a fairly fundamental trade-off.  For a solo node to get good performance, we need a high initial probability.  But a high initial probability allows the adversary to overload the system for long enough to prevent good throughput.  

The basic problem is that we have no ``admission control.'' The packets that join the system once the system is overloaded need to stay out of the way until the existing packets complete.

For example, consider the revised protocol where a packet remains ``passive'' until it sees either a silent round or a successful decoding event.

The technical problem we were having was that when a packet entered the system, it increased $N(t)$, but it also increased the other terms as well (due to the increase in contention caused by the new packet).  There are different solutions to this:
\begin{itemize}
    \item If the initial probability $p = e^{-\threshold}$, then the increase in contention is small, and so we still get the throughput we want.  Unfortunately, our analysis still causes trouble when there is only one packet in the system, as there is no time to amortize the cost.
    \item If a packet enters in a round with a successful decoding event, then we may be able to bound the increase according to the new packets with the decrease of the packets that finished.  But this seems easy to overwhelm.  For example, at most $\threshold$ packets finished, so if the adversary injects $2(1/p)\threshold$ packets, then the total contention increases.
    \item If a packet enters in a silent round, there is really nothing to counteract the increase in potential.  
\end{itemize}

With admission control, you could divide the execution into sub-executions: during each sub-execution, no new nodes enter the system, and potential decreases (in expectation) in each round.  At the beginning of each sub-execution, there is either a silent round or a successful decoding event, so (w.h.p.) the contention just before the new sub-execution began is $\leq \threshold$.  

If you modify the admission control to only enter on a silent round, we can then bound the contention at the beginning of each sub-execution: if $n$ nodes enter at the beginning of the sub-execution, the contention is $\leq O(1) + n + \threshold|\log(pn/\lambda)| + \threshold(\log(pn)/\pmin)$.  That is, there is $O(1)$ remaining potential from old packets, and then all the potential from the new packets.  

This helps, but fundamentally, you still cannot avoid some constant factor throughput as the best possible. Imagine, e.g., the adversary repeatedly injects $4\threshold$ packets and then waits until they finish before injecting more.  Assuming that takes even 5 epochs, you are getting a throughput of at best $80\%$. 


\subsection{New Potential Function}

To briefly summarize the algorithm, at the end of an epoch, packets multiplicatively update their broadcast probabilities by a factor of $\kappa^{1/4}$. Packets join with broadcast probability $p_0 = 1/\sqrt{\kappa}$. Inactive packets only become active upon hearing a silent epoch. Our desired contention is $c_* = \sqrt{\kappa}$. If contention is in the window $[\kappa^{1/4}, \kappa^{3/4}]$, we say that contention is good.

\[\Phi(t) = N(t) + logC(t) + s(t)\]

Our potential function $\Phi(t)$ has three components.

\begin{itemize}
    \item $N(t)$ is the total number of packets in the system at time $t$.
    
    \item $logC(t) = \alpha | \log_\kappa \frac{c(t)}{c_*} |$.
    
    fu
    
    \[
    \alpha = \begin{cases} 
    4\kappa, & \text{if } c(t) \geq c_* \\
    4, & \text{otherwise}.
    \end{cases}
    \]
    
    Foo
    
    This second term approximates the length of time until contention is near $c_*$. In the case that $c(t) > c_*$, we will have to wait roughly $4 \log_\kappa c(t) / c_*$ epochs of length $\kappa$ for good contention, as each epoch is of length $\kappa$ when more than $\kappa$ packets join, and most epochs result in noise, moving contention towards $c_*$ by a multiplicative factor of $\kappa^{1/4}$. In the case that $c(t) < c_*$, we will have to wait roughly $4 \log_\kappa c_*$ epochs of length $1$ for good contention.
    
    bar
    
    \item $s(t) = \alpha \log_\kappa \frac{c(t)}{c_*} + 4\kappa \log_\kappa \frac{1}{p_{min}}$. This third term accounts for decreases in contention due to successful packet transmissions.
\end{itemize}

\subsection{The Non-Arrival Case}

We consider three cases at the beginning of a given epoch: high contention, low contention, and good contention.


\paragraph{Case 1: \boldmath$c(t) > \kappa^{3/4}$.}

In this case of high contention, we will bound the probability of a silent epoch and then argue that both successful epochs and busy epochs result in a decrease to our potential function equal to the length of the epoch.

Applying Chernoff Bounds, the probability of a silent epoch is at most $1 / 2^{\Theta(\kappa^{3/4})$. A silent epoch will have no effect on the first and third terms, leaving them unchanged. It will, however, increase the second term by $\kappa$. Thus the expected increase in potential due to the possibility of a silent epoch is at most $\kappa / 2^{\Theta(\kappa^{3/4})}$.

Busy epochs leave the first and third terms unchanged, but result in a $\kappa$ decrease to the second term over $\kappa$ steps due to contention decreasing multiplicatively by a factor of $\kappa^{1/4}$.

Successful epochs effect all three terms. A successful epoch of length $u$ will result in a decrease to the first term of exactly $u$, as $u$ packets successfully transmitted. We now argue that the the combined effect of a successful epoch on the second and third terms is non-positive, as the third term is designed to decrease and counter any increase in the second term due to packet successes. Firstly, packet successes can only increase $p_min$, which would decrease our potential function. However, packet successes also decrease $c(t)$ to $c(t)'$, which does decrease the third term but may actually increase the second term if the newly resulting $\log_{\kappa} c(t) / c*$ is negative. This is the case where the increase in potential due to the second term is perfectly matched by the decrease in potential due to the third term. In particular, we note that the net change in the 3rd term is $\alpha \log_\kappa C($

\paragraph{Case 1: \boldmath$c(t) < \kappa^{1/4}$.}


\paragraph{Case 1: \boldmath$\kappa^{1/4} \leq c(t) \leq \kappa^{3/4}$.}


\subsection{Arrivals}

Suppose a batch of $m$ packets arrive at time $t$. By how much does this batch increase our potential function $\Phi(t)$?

The first term, $n(t)$, increases by $m$.

Contention increases from $c(t)$ to $c(t) + m/\sqrt{\kappa}$. The second term is only increasing with $c(t)$ when $c(t) > c_*$ (and the log is positive). Thus the increase in the second term is at most $4\kappa log_kappa \frac{c(t)$
